Grid computing is a form of distributed computing where clients donate computing
power to solve problems that to big to solve for a single machine. One of the
most famous grid computing projects is SETI@Home (Search for Extraterrestrial
Intelligence), in which PC users donate unused CPU cycles to find signs of
extraterrestrial life in signals from outer space. Such a grid consists of
participating clients and can be seen as a virtual supercomputer. The clients in
a grid are usually not connected to each other, but in stead communicate with a
master node or some other sort of scheduler.
	
Grid computing has several advantages compared to supercomputers. Since programs
run on normal computer hardware, no special software is needed. Also debugging
is a lot easier since this can be done on a single client. Grid computing
eliminates the complexity of shared memory and shared storage space, because
this approach forces the developers to think differently about their storage and
memory.

Grid computing projects such as SETI@Home currently require a specific program
that is installed on the PC of the user. However, this is rather intrusive, and
limits users of for example locked-down systems, that might otherwise be
interested to let their system be part of the grid. Why couldn't you run your
grid compute application in the browser, that can be found on everybody's PC?
Martin Remy proposed \cite{CasualGridComp} such an application, and we thought
it would be interesting to experiment with such an approach.
