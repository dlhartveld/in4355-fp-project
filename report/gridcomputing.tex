	Grid computing is a form of distributed computing where clients donate computing power to solve problems that to big to solve for a single machine.
	One of the most famous grid computing projects is SETI@Home (Search for Extraterrestrial Intelligence), in which PC users donate unused CPU cycles to
	find signs of extraterrestrial life in signals from outer space. The grid exists of all participating client and can be seen as a super virtual computer.
	The clients in a grid are mostly not connected to each other. 
	
	Grid computing has several advantages compared to super computers. Since programs run on normal pc hardware, no special software is needed.
	Also debugging is a lot easier since this can be done on a single client. Grid computing eliminates the complexity of shared memory and shared storage space.
	
	