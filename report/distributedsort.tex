\subsection{Overview}

An important aspect in MapReduce computations is the requirement that the output of a mapper
gets routed to the correct reducer.
Multiple mappers can output data with the same key and this data should end up
at the same reducer.
In MapReduce this is done by the partitioning function. This function determines
the reducer that should receive the data by doing a transformation from the key
to the index of a reducer. Reducers start fetching the data from the mappers and
then sort the received data locally.

\subsection{Sort out the sorting}

Our MapReduce implementation on the grid this gives some challenges.
Our clients do not have a shared file system where they can write data to, so
the reducers cannot fetch the data from the mappers.
A possible solution for this is to let the server gather the data and then
create reducer jobs of the gathered data.
In this case it would be important to create enough reduce jobs.

In our current implementation there is no partitioning function. The mappers
send the data sorted to the server which merges the sorted inputs
and then distributes the data to the reducers. The sort step is therefore
performed by the mappers and the server instead of by the reducers.
We understood the role of the partitioner only later in the development of the platform
when we already had the server sorting and therefore we chose not to change this.