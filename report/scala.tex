% string concatenation has horrible performance in a functional scenario.
% use stringbuilder instead, but this breaks side-effects-free coding. Monad?

For each of us, this project was our initial introduction to the Scala
language. Between the three of us, we had a lot of experience with Java and C\#.
We did have some knowledge of functional programming, but this was very limited.
Therefore, we had the opportunity to extensively discover and experience the
possibilities and limitations of functional programming in comparison to
object-oriented programming. This section describes some observations on the use
of Scala and its functional features in addition to the imperative,
object-oriented programming style, in comparison to imperative, object-oriented
programming in Java or C\#.

\subsection{Collection manipulation}
The most obvious advantage of functional language features such as higher-order
functions and lambda expressions is their use in the context of collections. The
availability of map, reduce, fold, foreach, filter, etc. makes it possible to
write very, very compact code that is highly expressive. What is even nicer, is that
you can parallelize function application through the use of parallel collections,
with the Scala standard libraries taking care of all the hard work.

However, not everything is as nice as the above statement might suggest. As soon
as you start to introduce side effects in the functions that are used as parameters
for e.g. map or foreach, things start to break down. For example, we wanted to
generate a long string representing a JSON object from a list of word counts. Where
the elegance of the code was pretty high when using simple string concatenation with '+',
it started to rot hard when we wanted to improve the performance of the code through
the use of a StringBuilder: The elegant one-liner lambda function turned into an ugly
imperative beast. It is precisely in these cases where the strain between the functional and
the imperative worlds is highest - respectively side-effect free programming versus
imperative programming.

\subsection{Option vs Exception}
Another example of a niceness of functional languages is the concept of Option[T].
Where, in most imperative, object-oriented programming languages, you have to throw
and catch exceptions, which litter through your code as soda cans on a high school playground,
the Scala language supports the Option[T] type that takes away some of this pain.
It is primarily useful in cases where you know that called code might behave in an
exceptional way (and where OO code would throw an exception), but where you are not
quite interested in that fact, but just in the result of the computation, which would
be nothing, or some default value, in the case of failure.

Scala supports this through the case class Option[T]. When a function returns an Option on T,
it simply means that, in case of succes, it is Some T, and in case of failure, it is
None (representing no result, nothing). In many cases, this is sufficient information
for the caller. For example, in the scala client implementation, we used the Apache HttpClient
library \cite{HttpClient} to connect to the server. The library would throw an exception if
the server could not be reached. However, in our client code, we connect to the server
to ask whether there is a job to be done. Here we are not interested in connection failures,
but just in the question of whether a job needs to be done or not. So we let the HttpClient
wrapper return an Option of HttpResponse, which is Some(HttpResponse) on succes, and None
on failure. A simple pattern matching on each of these conditions in the caller method is then 
enough to distinguish between these situations. While it is a specific solution, it does
demonstrate the elegance of the Option[T] pattern\footnote{In Haskell, the Maybe type is the equivalent of Scala's Option}.
