\subsection{Overview}
Our initial intention was to limit ourselves to the use of Fay \cite{Fay} in
stead of javascript. Fay is a programming language which contains the grammer
and syntax of the functional programming language Haskell, and some additional
grammer and syntax which enables you to store variables and interact with
javascript. Before you can use Fay, you have to install the Fay compiler. Once
that is installed you can compile Fay source files into Javascript source files
using the command line. 

\subsection{Problems}
The documentation on the Fay website seems to be
outdated, as the compiler arguments have changed. Luckily the Fay compiler can
tell you what the command line options are. However this was not our only
problem with Fay. Some of the compiler's features did not work as specified.
Supposedly it should be possible to format/beautify the produced Javascript code
using the command line argument "--pretty", but this doesn't seem to be doing
anything. The same applies to the "--library" argument which ensures that in the
resulting Javascript source code, the main function is not immediately called.

But most problematic of all was the fact that Fay produces Javascript code which
is wrapped in a function scope to avoid leaking scope. This means that we cannot
call any functions compiled by Fay from external Javascript source code. Even if
this scope problem was fixed, we'd still be unable to do this since the
generated Javascript functions have human-unreadable names like "$62$$62$$61$".
Because of this we've had to implement most of the client code dealing with the
communication and computation in Javascript in stead of Fay.
