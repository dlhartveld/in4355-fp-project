The functional programming style is supported more and more in programming
languages. Besides functional languages such as Scheme and Haskell, relatively
new languages like Javascript and Scala have features from functional
programming in their repetoire. Another example of this is LINQ (Language
Integrated Query) and its use of lambdas in C\#, but there are lots of other
non-functional languages who support some form of functional programming
constructs. To get familiar with functional programming principles and languages
we built a small grid computing framework to do MapReduce computations.
MapReduce is inspired by the map and reduce functions commonly used in
functional programming.

This article focuses on the functional aspects, details, pitfalls and
shortcomings of languages we came accross while building the grid computing
framework with functional languages. We will discuss the implementation of a
simple MapReduce algorithm and how we distributed the map and reduce steps over
several browser clients. We chose word count as the algorithm to implement,
because of its simplicity and the clearly defined operations for the map and
reduce steps.

The source code for the programs that we have built can be found in the following
two GitHub project repositories:
\begin{itemize}
  \item Server, and browser client (served by the server):\\
  		https://github.com/dlhartveld/in4355-fp-project
  \item The Scala wordcount client:\\
  		https://github.com/dlhartveld/in4355-grid-client-scala
\end{itemize}
