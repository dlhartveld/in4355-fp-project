Functional programming is increasingly used in programming languages. Besides new functional languages such as Scheme, Haskell and Scala, 
functional programming concepts are also added to existing non-functional languages. An example of this is LINQ (Language Integrated Query) 
and its use of lambdas in C\#, but there are lots of other non-functional languages who support some form of functional programming constructs.
To get familiar with functional programming principles and languages we built a small grid computing framework to do map-reduce computations.
MapReduce is in spired by the map and reduce functions commonly used in functional programming.

This article focuses on the functional aspects, details, pitfalls and shortcomings of languages we cam accross while building the grid computing framework
with functional languages. We will discuss the implementation of a simple map-reduce algorithm and how we distributed the map and reduce steps over connected clients. 
We chose word count as the algorithm to implement, because of its simplicity and the clearly defined operations for the map and reduce steps.