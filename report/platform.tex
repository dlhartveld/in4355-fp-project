	\subsection{Components}
	David
	\subsection{How does it work?}
		The server contains a list of jobs. To request a map or reduce job, a client can navigate to /jobs. 
		This call returns an id for the current job, that the client can execute. Using this id, 
		the client can request the javascript code for the job with the corresponding id by calling /{id}/code
		and the input data for the job with the corresponding id by calling /{id}/input.
		The code for the job should contain a call to the fetch function to request data and should contain a call to the push
		function to send the result back to the server.
		Data can be send to the server by calling /{id}/output.
		
		In this way, the framework is very generic meaning that a variety of MapReduce algorithms can be run on the platform.
		
		The browser client starts to check for pending tasks when the start button is clicked and stops when the stop button is clicked.
		When the stop button is clicked while executing a job, it will continue until the job is finished.